\chapter{Summary}
In this thesis, we aim at implementing the camera-based heart rate(HR) detection method under realistic situations. The whole paper are generally divided into two parts: the algorithm framework improvement and the guideline derived from self-collected datasets. 

For the first part, we improved the algorithm framework in three aspects: 1)the former Discriminative Response Map Fitting(DRMF) method for detecting facial landmarks are replaced by DLIB method, which is in higher time efficiency and accuracy; 2)the concept of rectified ROI is brought up and two-way tracking method is deployed to better eliminate subject's rigid head movement; 3)a new way to extract ground truth for our self-collected materials is proposed. 

The second part is based on the first section. With the updated algorithm framework, evaluations are conducted both on the recognized MAHNOB-HCI database and our self-collected materials under designed practical scenarios. The effectiveness of current framework is testified in the MAHNOB-HCI database experiment and a guideline for further studies concerning this topic is derived from the results of second evaluation. The guideline contains six realistic scenarios(head movement, illumination variation, speaking, glass wearing, making up and skin colors) and points out corresponding study direction for each scenario. With this guideline, we develop an innovative method in the view of head rotation scenario(reducing the measure error from 9.9 beats per minute to 5.43 beats per minute), which could also prove that our guideline is promising and inspiring in solving the problems of contact-free HR detection occurred in realistic situations.  
